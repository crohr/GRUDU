%****************************************************************************%
%* Annexes		                                                            *%
%*                                                                          *%
%* Author(s):                                                               *%
%* - Abdelkader AMAR (Abdelkader.Amar@ens-lyon.fr)                          *%
%* - David LOUREIRO (David.Loureiro@ens-lyon.fr)                            *%
%*                                                                          *%
%* $LICENSE$                                                                *%
%****************************************************************************%
%* $Id: GUM_annexes.tex,v 1.3 2007/11/29 16:03:21 dloureir Exp $
%* $Log: GUM_annexes.tex,v $
%* Revision 1.3  2007/11/29 16:03:21  dloureir
%* typo corrections
%*
%* Revision 1.2  2007/11/08 11:31:14  dloureir
%* Correcting the headers
%*
%****************************************************************************%
\appendix
\chapter{Configuration files}

\section{Files used by \grudu}

\grudu uses several configuration files saved in the \texttt{.diet}
directory which is located at the root of your home directory:

\begin{itemize}
  \item \texttt{GRUDUApplicationProperties.xml} : main configuration file containing
  the high level information.
  \item \texttt{g5K.xml} : this file contains the main information for the
  \gfk connection management.
  \item \texttt{g5k\_cfg.xml} : this file corresponds to the grid description.
\end{itemize}

For each file the main content for the \grudu usage will be described.

\section{\gfk configuration file: \textit{GRUDUApplicationProperties.xml}}
\begin{verbatim}
<application>
  <properties 
        name="tipOfTheDayShowOnStartup" 
        value="false" />
  <properties 
        name="tipOfTheDayFileOfTips" 
        value="languages/totd/defaultGRUDUFileOfTips_eng.xml" />
  <properties 
        name="version" 
        value="1.1.0" />
</application>
\end{verbatim}

This files defines the application wide configuration iformation. 

For the moment every elements are a \texttt{properties} with a \texttt{name} and
a \texttt{value}.

For the moment there are three properties declared in that file :
\begin{itemize}
  \item \textit{version} : the purpose of that property is obvious, it
  corresponds to the version of \grudu.
  \item \textit{tipOfTheDayShowOnStartup} : This property declares whether the
  tip of the day should (or not) be displayed on startup.
  \item \textit{tipOfTheDayFileOfTips} : This property defines the file from
  which the tips of the day should be taken.
\end{itemize}

\section{\gfk configuration file: \textit{g5k.xml}}

\begin{verbatim}
<?xml version="1.0" standalone="yes"?>
<g5k>
  <preferredAccesPoint host="acces.lyon.grid5000.fr.fr" />
  <username id="myG5KLogin" />
  <sshkey file="thePathToMySSHKeyFile" />

<!-- G5K Sites -->
 . . .
</g5k>

\end{verbatim}

This file defines the global information used to log in \gfk. The elements that
are not used by \grudu have been removed from the description.

Here are the parameters that should be supplied:

\begin{itemize}
  \item \textit{preferredAccesPoint}: the node has an attribute named
  \textbf{host}. This attribute have to be the name of one of the access
  frontales of \gfk.
  \item \textit{username}: the node has an attribute named \textbf{id} that is
  the login of the user.
  \item \textit{sshkey}: the node has one attribute \textbf{file} which is the
  file storing the ssh private key.
\end{itemize}

\section{\gfk configuration file: \textit{g5k\_cfg.xml}}

\begin{verbatim}

<g5k>
        <site
                 id="Lyon"
                 enable="false"
                 batch_schedulers="OAR1"
         >
                <cluster name="Lyon--Capricorne"  xda="" />
                <cluster name="Lyon--Sagittaire"  xda="" />
        </site>

. . .

</g5k>

\end{verbatim}

This file describes the platform of \gfk, the properties of the sites
(id, batch\_scheduler, etc) but also the clusters of the sites with their
deployment partitions.

Here are the parameters that should be supplied for each site:

\begin{itemize}
  \item \textit{id} : Name of the site
  \item \textit{enable} : (true/false) defines whether the site is considered in
  the interrogation parts of \grudu
  \item \textit{batch\_schedulers} : Name of the batch scheduler to use
  \item For each cluster:
  \begin{itemize}
    \item \textit{name} : the name of the cluster
    \item \textit{xda} : the partition used by KaDeploy
    \end{itemize}
\end{itemize}

%******************************************%